\documentclass[a4paper,12pt]{article}
\usepackage[a4paper,top=1.3cm,bottom=2cm,left=1.5cm,right=1.5cm,marginparwidth=0.75cm]{geometry}
\usepackage{setspace}
\usepackage{cmap}					
\usepackage{mathtext} 				
\usepackage[T2A]{fontenc}			
\usepackage[utf8]{inputenc}			
\usepackage[english,russian]{babel}
\usepackage{multirow}
\usepackage{graphicx}
\usepackage{wrapfig}
\usepackage{tabularx}
\usepackage{float}
\usepackage{longtable}
\usepackage{hyperref}
\hypersetup{pdfstartview=FitH,  linkcolor=linkcolor,urlcolor=urlcolor, colorlinks=true}
\hypersetup{
    colorlinks=true,
    linkcolor=blue,
    filecolor=magenta,      
    urlcolor=cyan,
    pdftitle={Overleaf Example},
    pdfpagemode=FullScreen,
    }
\usepackage[rgb]{xcolor}
\usepackage{amsmath,amsfonts,amssymb,amsthm,mathtools} 
\usepackage{icomma} 
\mathtoolsset{showonlyrefs=true}
\usepackage{euscript}
\usepackage{mathrsfs}
\usepackage{ dsfont }
\usepackage{fancyhdr}
\linespread{1.25} 
\date{}
\DeclareMathOperator{\rg}{rg}
\DeclareMathOperator{\im}{Im}
\DeclareMathOperator{\Ker}{Ker}
\DeclareMathOperator{\tr}{tr}
\DeclareMathOperator{\det}{det}
\DeclareMathOperator{\log}{log}
\DeclareMathOperator{\tg}{tg}
\DeclareMathOperator{\arcsin}{arcsin}
\DeclareMathOperator{\ctg}{ctg}
\DeclareMathOperator{\arccos}{arccos}
\usepackage{bigints}


\title{Вступительное тестирование школьников. ПГУ 2023.}

\begin{document}

\maketitle
\thispagestyle{empty}
\large{
\noindent
a. Посчитайте без калькулятора: $\sqrt[5]{\frac{0.(835) + 0.(005) - 0.01\cdot|5\cdot(5^2-5\cdot 6)|}{\frac{787}{1332}}}$.\\\\
b. Упростите выражение:
$\sqrt[3]{1-3\sqrt[3]{18}+3\sqrt[3]{12}}$.\\\\
c. Сравните числа: $2^{3^{100}} \text{и}\  3^{2^{150}}$.\\\\
d. Вычислите до 3 знака после запятой выражения: $\sqrt[3]{-3}$ и $(-3)^\frac{1}{3}$.\\\\
e. Найдите $f(x)$, если $2f(x+2)+f(4-x) = 2x+5.$\\\\
f. Докажите неравенство: $\log_{2021}(2023) > \frac{\log_{2021}(1) + \log_{2021}(2)+\ \dotsc \ +\log_{2021}(2022)}{2022}$.\\\\
g. Упростите выражение: $\tg(\arcsin(x)) - \ctg(\arccos(x))$.\\\\
h. Найдите все такие пары квадратных трёхчленов $x^2+ax+b,\ x^2+cx+d$, что $a$ и $b$ -- корни второго трёхчлена, $c$ и $d$ -- корни первого.\\\\
i. Существует ли треугольник, у которого сумма косинусов внутренних углов равна 1?\\\\
j. Выведите формулу длины эллипса с любой удовлетворяющей точностью.\\\\
k. Выведите период гармонических колебаний для математического маятника без учёта малости угла отклонения, то есть без приближения $\sin{\alpha} \approx \alpha$.\\\\
l. Вычислите следующие неопределённые интегралы:\\
$\bigints\frac{(1+x^2)\cdot \arcsin(x)}{x^2\cdot \sqrt{1-x^2}} \,dx$, $\bigints \frac{2x^2-3x}{\sqrt{x^2-2x+5}} \,dx$, $\bigints \frac{x^4+2x^2+4}{(x^2+1)^3} \,dx$.
}

\end{document}
