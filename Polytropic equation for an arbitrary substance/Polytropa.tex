\documentclass[a4paper,12pt]{article}
\usepackage[a4paper,top=1.5cm,bottom=2cm,left=1.5cm,right=1.5cm,marginparwidth=0.75cm]{geometry}
\usepackage{setspace}
\usepackage{cmap}					
\usepackage{mathtext} 				
\usepackage[T2A]{fontenc}			
\usepackage[utf8]{inputenc}			
\usepackage[english,russian]{babel}
\usepackage{multirow}
\usepackage{graphicx}
\usepackage{wrapfig}
\usepackage{tabularx}
\usepackage{float}
\usepackage{longtable}
\usepackage{hyperref}
\hypersetup{colorlinks=true,urlcolor=blue}
\usepackage[rgb]{xcolor}
\usepackage{amsmath,amsfonts,amssymb,amsthm,mathtools} 
\usepackage{icomma} 
\mathtoolsset{showonlyrefs=true}
\usepackage{euscript}
\usepackage{mathrsfs}



\begin{document}
    \newpage
    \begin{spacing}{1.8}
        \centering{\Large{\textbf{Уравнение политропы $c = const$ для произвольного вещества. Связь изотермической и адиабатической сжимаемостей.}}}
    \end{spacing}
    
    \large{Оговорим, что нам известна зависимость $T(p, V)$.}
    \section*{1) Нахождение $C_{\alpha} - C_{V}$:}

    Первое начало термодинамики:
    \[ C_{\alpha}dT = dU + pdV \]
    \[ C_{\alpha}dT = \left(\frac{\partial U}{\partial T} \right)_{V}dT  + \left(\frac{\partial U}{\partial V} \right)_{T}dV  + pdV\]
    \[ C_{\alpha}dT = \left(\frac{\partial U}{\partial T} \right)_{V}dT + \left(\left(\frac{\partial U }{\partial V}\right)_{T} + p\right)dV \]
    Крайне очевидно, что $\left(\frac{\partial U}{\partial T}\right)_{V} = C_{V}$, тогда: 
    \[ C_{\alpha} - C_{V} = \left(\left(\frac{\partial U }{\partial V}\right)_{T} + p\right)\left(\frac{\partial V}{\partial T}\right)_{\alpha}\]

    \section*{2) Аналогично выводу политропы для идеального газа 
    найдём $\frac{C_{\alpha} - C_{V}}{C_{p} - C_{V}}$:}
    \[ \frac{C_{\alpha} - C_{V}}{C_{p} - C_{V}} = \frac{T\left(\frac{\partial p}{\partial T}\right)_{V}\left(\frac{\partial V}{\partial T}\right)_{\alpha}}{T\left(\frac{\partial p}{\partial T}\right)_{V}\left(\frac{\partial V}{\partial T}\right)_{p}} \]

    Немного пошаманим над уравнением выше:
    \[ dT = \left(\frac{C_{p} - C_{V}}{C_{\alpha} - C_{V}}\right)\left(\frac{\partial T}{\partial V}\right)_{p}dV \]

    Напоминаю, что нам по условию дана зависимость $T = T(p, V)$, то есть:
    \[ dT = \left(\frac{\partial T}{\partial p}\right)_{V}dp + \left(\frac{\partial T}{\partial V}\right)_{p}dV \]
    
    \newpage

    Приравниваем $dT$ и получаем по заслугам:
    \[ \left(\frac{C_{p} - C_{V}}{C_{\alpha} - C_{V}}\right)\left(\frac{\partial T}{\partial V}\right)_{p}dV = \left(\frac{\partial T}{\partial p}\right)_{V}dp + \left(\frac{\partial T}{\partial V}\right)_{p}dV \]

    \[ \left(\frac{\partial T}{\partial p}\right)_{V}dp + \left(1 - \frac{C_{p} - C_{V}}{C_{\alpha} - C_{V}}\right)\left(\frac{\partial T}{\partial V}\right)_{p}dV = 0 \]

    И, не поверите, находим уравнение политропы в общем случае:
    \[ \left(\frac{\partial T}{\partial p}\right)_{V}dp + \left(\frac{C_{\alpha} - C_{p}}{C_{\alpha} - C_{V}}\right)\left(\frac{\partial T}{\partial V}\right)_{p}dV = 0 \]

    \section*{3) Найдем связь между термической $\beta_{T}$ и адиабатической $\beta_{S}$ сжимаемостями:}
    Для начала $\beta_{T} = -\frac{1}{V}\left(\frac{\partial V}{\partial p}\right)_{T}$ и $\beta_{S} = -\frac{1}{V}\left(\frac{\partial V}{\partial p}\right)_{S}.$
    \newline
    Из свойств частных производных имеем соотношение 
    $\left(\frac{\partial V}{\partial p}\right)_{T}\left(\frac{\partial p}{\partial T}\right)_{V}\left(\frac{\partial T}{\partial V}\right)_{p} = -1.$
    Из выведенного ранее уравнения политропы получаем уравнение адиабаты:
    \[ \left(\frac{\partial T}{\partial p}\right)_{V}dp + \frac{C_{p}}{C_{V}}\left(\frac{\partial T}{\partial V}\right)_{p}dV = 0 \]
    Выразим $\left(\frac{\partial V}{\partial p}\right)_{T}.$  Получим $\left(\frac{\partial V}{\partial p}\right)_{T} = -V\beta_{T}$. Подставим это в соотношение соотношение связи между собой частных производных, тогда:
    \[ \left(\frac{\partial V}{\partial T}\right)_{p}\left(\frac{\partial T}{\partial p}\right)_{V} = V\beta_{T}.\]

    Из уравнения адиабаты получаем:
    \[ \left(\frac{\partial T}{\partial p}\right)_{V} + \frac{C_{p}}{C_{V}}\left(\frac{\partial T}{\partial V}\right)_{p}\left(\frac{\partial V}{\partial p}\right)_{S} = 0 \]

    \[ \left(\frac{\partial T}{\partial p}\right)_{V} = -\frac{C_{p}}{C_{V}}\left(\frac{\partial T}{\partial V}\right)_{p}\left(\frac{\partial V}{\partial p}\right)_{S} \]

    \[ \left(\frac{\partial T}{\partial p}\right)_{V} = \frac{C_{p}}{C_{V}}\left(\frac{\partial T}{\partial V}\right)_{p}V\beta_{S}\]

    \[ \left(\frac{\partial T}{\partial p}\right)_{V}\left(\frac{\partial V}{\partial T}\right)_{p} = \frac{C_{p}}{C_{V}}V\beta_{S}\]

    \newpage
    Заметим, что левая часть - в точности $V\beta_{T}$. Ну а тогда:
    \[ V\beta_{T} = \frac{C_p}{C_V}V\beta_{S} \]

    \[\beta_{T} = \frac{C_p}{C_V}\beta_{S} \]
    
    
\end{document}